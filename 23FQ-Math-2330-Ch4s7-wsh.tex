%\documentclass[10pt]{article}
%%\input{../../common/1021_header.tex}
%\input{../../common/header_old.tex}
%\usepackage{tikz, wasysym, framed}
%
%\newcommand{\ihat}{\boldsymbol{\hat{\textbf{\i}}}}
%\newcommand{\jhat}{\boldsymbol{\hat{\textbf{\j}}}}
%\newcommand{\khat}{\boldsymbol{\hat{\textbf{k}}}}
%
%\let\oldvec\vec
%\renewcommand{\vec}[1]{\oldvec{\mathbf{#1}}}
%
%\newcommand{\uhat}{\boldsymbol{\hat{\textbf{u}}}}
%\newcommand{\grad}{\vec{\nabla}}
%\newcommand{\<}{\left\langle}
%\renewcommand{\>}{\right\rangle}
%
%
%\begin{document}
% Instructions to change to html version:
% Comment out:
%  minipage, multicols,columnbreak, mathbf, hrule
% Replace all:% %%\begin{minipage}% %%%%%\end{minipage} %%%%%%\begin{mulicols}  %%%%%%\end{mulicols}  %%%%%\columnbreak % %%%%%%\begin{framed} %%%%%\endframed} %%%%%%\hrule
% Search for \mathbf
% Replace \\] with \[ and \) with \(
% Enclose graphics in figure environments and add captions
% Re-tag \df environments as sections, subsections, etc.
% Command Line Code to Create html version:
%First: pdflatex -shell-escape filename.tex                                   
%Second, for each figure: inkscape "filename-figure1.pdf" -o "filename-figure1.png"
% Third: htlatex filename.tex "ht5mjlatex.cfg, charset=utf-8" " -cunihtf -utf8"


\documentclass[10pt]{article}

%\usepackage{tikz, pgf,pgfplots,wasysym,array}
%\usepackage{wasysym,array}

\usepackage{amsmath,amssymb}

\ifdefined\HCode
  \def\pgfsysdriver{pgfsys-tex4ht-updated.def}
\fi 
%\ifdefined\HCode
%  \def\pgfsysdriver{pgfsys-dvisvgm4ht.def}
%\fi 
\usepackage{tikz}
\usetikzlibrary{calc,decorations.markings,arrows}
\usepackage{pgfplots}

\pgfplotsset{compat=1.12}
\usepackage{myexternalize}
\usetikzlibrary{calc,decorations.markings,arrows}
\usepackage{framed}
\usepackage[none]{hyphenat}

\input{../../../common/1336_header_test.tex}


\newcommand{\ihat}{\boldsymbol{\hat{\textbf{\i}}}}
\newcommand{\jhat}{\boldsymbol{\hat{\textbf{\j}}}}
\newcommand{\khat}{\boldsymbol{\hat{\textbf{k}}}}

%\let\oldvec\vec
%\renewcommand{\vec}[1]{\oldvec{\mathbf{#1}}}

\newcommand{\uhat}{\boldsymbol{\hat{\textbf{u}}}}
\newcommand{\grad}{\vec{\nabla}}
\newcommand{\<}{\left\langle}
\renewcommand{\>}{\right\rangle}


\begin{document}


\renewcommand{\myTitle}{MATH 2330: Multivariable Calculus}

\renewcommand{\mySubTitle}{4.7: Maximum \& Minimum Values}
%~\hfill Name: \underline{~~~~~~~~~~~~~~~~~~~~~~~~~~~~~~~~~~~~~~~~~~~~~~~}

\lectTitle{\vspace*{-.5in}\myTitle}{\vspace*{.1in}\mySubTitle \vspace*{-.25in}}



%\vspace*{-.5in}
\section*{Section 4.7 -  Maximum \& Minimum Values}

%\hspace*{-.8in}%%\begin{minipage}{1.25\textwidth}
%\begin{framed}

\subsection*{Types of Critical Points:}
\begin{multicols}{3}
\begin{itemize}
\item Local Minimum
\item Local Maximum
\item Saddle Point
\end{itemize}
\end{multicols}

%\hrule
\vspace*{.1in}

\subsection*{First Derivative Test:}
%\begin{multicols}{2}
A point \((a,b)\) is a \textbf{critical point} of \(f\) if:
\[f_x(a,b) = 0 \quad \text{\bf and} \quad f_y(a,b)=0\]
or if one of the partial derivatives does not exist at \((a,b)\).\\

\textbf{Geometric Interpretation:} If there is a local max or min where the tangent plane is defined, the tangent plane will be \textit{horizontal}.\\


%\hrule
\vspace*{.1in}


\subsection*{Second Derivative Test:}
In order to classify critical points, we calculate the quantity \(D\):
\[
D = \left|\begin{matrix}
f_{xx} & f_{xy}\\
f_{yx} & f_{yy}\\
\end{matrix}\right|
= f_{xx} f_{yy} - \left(f_{xy}\right)^2
\]
There are four possible outcomes from the 2nd derivative test, when \(D\) has been evaluated at the critical point \((a,b)\):
\begin{enumerate}[{Case} 1:]
\item \(D(a,b) > 0, \text{ and } f_{xx}(a,b) > 0\): Local Minimum at \((a,b)\)
\item \(D(a,b) > 0, \text{ and } f_{xx}(a,b) < 0\): Local Maximum at \((a,b)\)
\item \(D(a,b) < 0\): Saddle Point at \((a,b)\)
\item \(D(a,b) =0\): Inconclusive, Test Fails \(\Rightarrow\) try something else!
\end{enumerate}


%\end{multicols}
%
%
%%\end{framed}
%
%%%%\end{minipage}
%
%
%
%%\hspace*{-.8in}%%\begin{minipage}{1.25\textwidth}
%%\begin{framed}

%\hrule
\vspace*{.1in}

\subsection*{Global (Absolute) Maximum \& Minimum Values:}
%If we are interested in a \textit{closed} set of values for our domain, we must \textit{check the boundary as well}.
%
%
%
%
%\df{\textcolor{sblack}{The Extreme Value Theorem:}}~\\
%Guarantees that there is an \textbf{absolute (global)} minimum value and an \textbf{absolute (global)} maximum value at some point in a closed domain.
%
%\df{\textcolor{sblack}{Strategy for Finding Global Max/Min Values:}}~\\
To find the absolute max/min values on a closed region \(D\), assemble the candidates, then choose the largest/smallest:
\begin{enumerate}
\item Find the local maxima \& minima inside the region \(D\).
\item Find extreme values on the boundary of \(D\).
\item The largest value is the \textbf{global maximum} and the smallest value is the \textbf{global minimum}.
\end{enumerate}


%\end{multicols}


%\end{framed}

%%%\end{minipage}

\pagebreak

\section*{Example Problems:}


\begin{enumerate}[{Example} 1: ]
%\addtocounter{enumi}{1}
\item Find and classify all of the critical points of \(f(x,y) = 2x^3+y^3-3x^2-12x-3y\)

\vfill

\item Find and classify all of the critical points of \(f(x,y) = x^2y^4\)

\vfill

\item Find and classify all of the critical points of \(f(x,y) = x^2+4xy+3y^2-6x-12y\)

\vfill

%
%
%\end{enumerate}
%
%\begin{enumerate}[{Example} 1: ]
%\addtocounter{enumi}{3}

\item Find the global maximum and minimum values of \(f(x,y) = x^2y^4\) on the domain \(D=\left\lbrace x^2+y^2\leq 1\right\rbrace\).

\vfill

\end{enumerate}

\end{document}

